For = es una expresión de bucle con sintaxis similar al if pero un poco más compleja ya que esta declara una variable con un valor, una condición y un incremento, mientras las condiciones dentro del bucle sean true este se seguirá ejecutando 

&&= Este operador se llama (Y), se usa para evaluar dos expresiones y devuelve true solo si ambas son true 
 
||=Este operador se llama (O) devolverá true si una de las expresiones evaluadas es true 

!=Este operador se llama (NO) y se usa para convertir un valor booleano a su opuesto 